\documentclass[a4paper, 12pt]{scrartcl}


\usepackage[utf8]{inputenc}
\usepackage[T1]{fontenc}
\usepackage{lmodern}
\usepackage{listings}
\usepackage{graphicx}
\usepackage{amsmath}
\usepackage{amsfonts}
\usepackage{amssymb}
\usepackage{caption}
\usepackage{subcaption}
\usepackage[usenames,dvipsnames]{xcolor}


\setcounter{secnumdepth}{4}
% TAILLE DES PAGES (A4 serré)

\setlength{\parindent}{0pt}
\setlength{\parskip}{1ex}
\setlength{\textwidth}{17cm}
\setlength{\textheight}{24cm}
\setlength{\oddsidemargin}{-.7cm}
\setlength{\evensidemargin}{-.7cm}
\setlength{\topmargin}{-.5in}

% Commandes de mise en page
\newcommand{\fichier}[1]{\emph{#1}}
\newcommand{\nom}[1]{\emph{#1}}
\newcommand{\Fig}[1]{Fig \ref{#1} p. \pageref{#1}}
\newcommand{\itemi}{\item[$\bullet$]}

% Commandes de maths
\newcommand{\fonction}[3]{#1 : #2 \to #3}
\newcommand{\intr}[2]{\left[ #1 ; #2 \right]}
\newcommand{\intn}[2]{\left[\![ #1 ; #2 \right]\!]}
\newcommand{\intro}[2]{\left] #1 ; #2 \right[}
\newcommand{\intrsod}[2]{\left[ #1 ; #2 \right[}
\newcommand{\ps}[2]{\langle #1, #2 \rangle}
\newcommand{\mdelta}[1]{\boldsymbol{\delta_{#1}}}
%% \newcommand{\mdelta}[1]{\delta_{\textbf{#1}}}

\pagenumbering{arabic}
\graphicspath{{images/}}

\title{Rapport du projet de spécialité} 
\subtitle{Contribution au logiciel libre Git} 
\author{Antoine Delaite \and Louis--Alexandre Stuber \and Guillaume Pages \and Rémi Galan-Alphonso \and Rémi Lespinet}
\date{}

\begin{document}

\maketitle

\section{Introdution}

Le projet de spécialité a été un challenge intéressant. Les quatres semaines ont été intenses, et le groupe a connu une évolution dans l'organisation et la communication durant cette période. 

\section{Présentation du logiciel Git}

\subsection{Le logiciel git}

Git est un logiciel de versionnage très populaire parmi les developpeurs.

\subsection{Les contraintes}


\section{Nos contributions}

\subsection{toto}

\section{La communauté Git}

\subsection{Junio, Jeff and Matt}

\section{La communauté Git}

\subsection{Junio, Jeff and Matt}

\section{Enseignements tirés}

\subsection{Enseignements techniques}

\subsection{Agilité}

\section{Conclusion}


\end{document}